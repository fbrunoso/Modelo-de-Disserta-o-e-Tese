%
% Documento: Resumo (Português)
%


\begin{center}
\imprimirtitulo
\end{center}

\begin{resumo}
Síntese do trabalho em texto cursivo contendo um único parágrafo. O resumo é a apresentação clara, concisa e seletiva do trabalho.
No resumo deve-se incluir, preferencialmente, nesta ordem: brevíssima introdução ao assunto do trabalho de pesquisa (qualificando-o quanto à sua natureza), o que será feito no trabalho (objetivos), como ele será desenvolvido (metodologia), quais serão os principais resultados e conclusões esperadas, bem como qual será o seu valor no contexto acadêmico. Para o projeto de dissertação e teses sugere-se que o resumo contenha de 150 a 500 palavras, de acordo com as normas da UESC \cite{normasuesc}. Caso esteja utilizando este modelo em outra instituição, observe as normas da mesma para o resumo.

\textbf{Palavras-chave}: latex. abntex. modelo.
 \textit{(Entre 3 a 6 palavras ou termos, separados por ponto, descritores do trabalho. As palavras-chaves são Utilizadas para indexação).}

\end{resumo}
