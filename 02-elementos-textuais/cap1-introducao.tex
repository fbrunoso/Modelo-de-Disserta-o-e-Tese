%
% Documento: Introdução
%

\chapter{Introdução}\label{chap:introducao}
Este template em \LaTeX \ de trabalho de final de curso foi elaborado para o programa de Pós-Graduação em Modelagem Computacional em Ciência e Tecnologia, mas também pode ser utilizado por outros programas de graduação e pós-graduação da UESC e até de outras instituições, caso se adéque as regras da mesma.

Para tanto, elaboramos a classe {\ttfamily ppgmc-uesc.cls}, construída com base nas normas da ABNT e da UESC para trabalhos de conclusão de curso. Maiores detalhes relacionados aos comandos existentes no estilo podrão da ABNT devem ser adquiridos através da documentação disponível no site \href{http://www.abntex.net.br/}{http://www.abntex.net.br/}. Encorajamos a consultar as regras do pacote abntex  \href{http://linorg.usp.br/CTAN/macros/latex/contrib/abntex2/doc/abntex2.pdf}{neste link}.

Para melhor entendimento do uso do estilo de formatação, aconselha-se que o  usuário analise os comandos existentes no arquivo {\ttfamily PPGMC.tex} e os resultados obtidos após copilação.

Os arquivos pré-textuais, textuais e pós-textuais podem ser modificados a vontade. Nas pastas contém apenas exemplos. As referências citadas também são apenas exemplos, não se referindo necessariamente as informações anunciadas. O arquivo principal {\ttfamily PPGMC.tex} também pode ser modificado ou renomeado.

A utilização desse template requer conhecimento prévio de Latex. A primeira versão foi lançada em 2015 para uso em editores offline. Agora estamos lançado este template online na plataforma \href{https://pt.overleaf.com}{\ttfamily \textit{overleaf}}. Sugestões de como melhorá-la e indicação de possíveis correções serão muito bem recebidas. A partir do segundo capítulo descrevemos como utilizar este template. Na seção \ref{sec:contatos} estão os contatos do autor para dúvidas e contribuições.

\section{Motivação}
\label{sec:motivacao}

Tendo em vista que o programa de Pós-Graduação em Modelagem Computacional em Ciência e Tecnologia ainda não tinha um estilo padrão para teses e dissertações, nem mesmo a UESC, resolveu desenvolver um template de padronização dos trabalhos do programa e demais programas de mestrado e doutorado da UESC. Um modelo de TCC da UESC pode ser encontrado \href{https://www.overleaf.com/latex/templates/modelo-tcc-uesc/sqtswzxtgwkj}{neste link}. Nada impede que este modelo seja utilizado também por programas de outras universidades, caso as regras sejam as mesmas ou possam ser modificadas nos arquivos aqui contidos.

O estilo de documento utilizado é o {\ttfamily abntex2}. Através desse estilo a constituição do documento torna-se facilitada, uma vez que o mesmo possui comandos especiais para auxiliar a distribuição/definição das diversas partes constituintes do projeto. 


Uma das principais vantagens do uso do estilo de formatação para LATEX é a formatação \textit{automática} dos elementos que compõem um documento acadêmico, tais como capa, folha de rosto, dedicatória, agradecimentos, epígrafe, resumo, abstract, listas de figuras, tabelas, siglas e símbolos, sumário, capítulos, referências etc.


