\documentclass[oneside]{pacotes/ppgmc-uesc} 	% imprimir apenas frente
%\documentclass[doubleside]{pacotes/ppgmc-uesc}	% imprimir frente e verso

% importações de pacotes
	\usepackage[alf, abnt-emphasize=bf, bibjustif, recuo=0cm, abnt-etal-cite=2, abnt-etal-list=0]{abntex2cite}	% citações padrão ABNT
	\usepackage{bookmark}					% cria menu de bookmarks
	\usepackage[utf8]{inputenc}				% acentuação direta
	\usepackage[T1]{fontenc}				% codificação da fonte em 8 bits
	\usepackage{graphicx}					% inserir figuras
	\usepackage{float}				    	%PERMITE usar H em figura
	%\usepackage{picinpar}					% imagens ao lado do texto
	\usepackage{amsfonts, amssymb, amsmath}		% fonte e simbolos matemáticos
	\usepackage{cancel}                     %cancelar termos numa expressão matemática
	\usepackage{lastpage}			% Usado pela Ficha catalográfica
	\usepackage{verbatim}					% texto é interpretado como escrito no documento
	\usepackage{multirow, array}				% múltiplas linhas e colunas em     tabelas
	\usepackage{indentfirst}				% indenta o primeiro parágrafo de cada seção.
	\usepackage[algoruled, portuguese]{algorithm2e}		% escrever algoritmos

	%\usepackage{times}				    	% usa a fonte Times
	\usepackage{hyperref} 		            % inserindo links azuis, referências em preto
	\usepackage{pacotes/subfigureppgmc}					% posicionamento de figuras
	\usepackage{multicol}                   % para usar o ambiente multicols-multiplas colunas
%	\usepackage{paralist}                  %permite listas especiais
	\usepackage{xcolor, colortbl}			% comandos de cores
	%\usepackage{breakurl}					% permite quebra de linha em urls
	
	%\usepackage{subeqnarray}				% sub enumeração de equações
	%\usepackage{makeidx}					% produzir índice remissivo (glossario)
	%\usepackage{multind}					% produzir índices múltiplos
	\usepackage{pdfpages}               % inserir arq. pdf no texto (ex.: folha de aprovação e ficha catalográfica)
	\usepackage{pdflscape}                % permite colocar página em formato paisagem
%	\usepackage[normalem]{ulem} % para o underline colorido

\makeatother
